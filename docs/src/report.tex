\documentclass[a4paper,12pt]{ctexart}
\usepackage{geometry}
\usepackage{graphicx}
\usepackage{amsmath}
\usepackage{float}
\usepackage{caption}
\usepackage{hyperref}
\usepackage{listings}
\usepackage{xcolor}



\lstset{
    language=Python,
    basicstyle=\small\ttfamily,
    breaklines=true,
    breakatwhitespace=false,
    showspaces=false,
    showstringspaces=false,
    frame=single,
    numbers=left,
    numberstyle=\tiny\color{gray},
    keywordstyle=\color{blue},
    commentstyle=\color[rgb]{0,0.6,0},
    stringstyle=\color[rgb]{0.58,0,0.82},
    extendedchars=false % 解决中文乱码问题(如果有)
}

% 设置页面边距
\geometry{left=2.5cm,right=2.5cm,top=2.5cm,bottom=2.5cm}

% 图片路径
\graphicspath{{../../images/}}

% 设置章节标题左对齐
\ctexset{
  section = {
    format = \Large\bfseries
  }
}

\begin{document}

% --- 封面 ---
\begin{titlepage}
    \centering
    \vspace*{2cm}
    
    % 标题
    {\ifx\heiti\undefined\bfseries\else\heiti\fi\fontsize{36pt}{48pt}\selectfont 信号处理综合课程设计\par}
    \vspace{0.5cm}
    {\ifx\heiti\undefined\bfseries\else\heiti\fi\fontsize{36pt}{48pt}\selectfont 说\hspace{1cm}明\hspace{1cm}书\par}
    
    \vspace{5cm}
    
    % 信息字段
    {\Large \songti
    \renewcommand{\arraystretch}{2} % 行高
    \begin{tabular}{lc lc}
        \textbf{设计题目:} & \multicolumn{3}{l}{\underline{\makebox[12.5cm][c]{\resizebox{12cm}{!}{基于 NI ELVIS III 的连续时间信号分析与采样系统设计}}}} \\
        \textbf{姓名:} & \underline{\makebox[5cm]{}} & \textbf{学号:} & \underline{\makebox[5cm]{}} \\
        \textbf{班级:} & \underline{\makebox[5cm]{}} & \textbf{成绩:} & \underline{\makebox[5cm]{}} \\
    \end{tabular}
    }
    
    \vfill
    
    {\large \today}
    \vspace{2cm}
\end{titlepage}

% --- 目录 ---
\newpage
\tableofcontents
\newpage

% --- 正文 ---
% --- 正文 ---
% 这是由 Pandoc 自动生成的正文内容
% 请不要直接修改此文件,而是修改对应的 content.md 文件

\section{设计内容与要求}

\subsection{设计内容}

本课程设计基于 NI ELVIS III 实验平台,主要包含以下三个核心实验内容:

\begin{enumerate}
\def\labelenumi{\arabic{enumi}.}
\item
  \textbf{傅里叶级数与滤波器实验}:利用函数发生器产生周期方波信号,通过可调低通滤波器(TUNEABLE
  LPF),观测不同截止频率下的输出波形,验证周期信号的分解与合成原理
  。
\item
  \textbf{连续时间系统时域分析}:构建一阶 RC
  电路,分别输入方波(近似阶跃信号)和窄脉冲(近似冲激信号),观测并分析一阶系统的阶跃响应与冲激响应
  。
\item
  \textbf{采样与混叠实验}:利用乘法器(MULTIPLIER)模拟信号采样过程,改变采样频率(\(4000\text{Hz}, 2000\text{Hz}, 500\text{Hz}\)),验证奈奎斯特采样定理并观测混叠现象
  。
\end{enumerate}

\subsection{设计要求}

\begin{enumerate}
\def\labelenumi{\arabic{enumi}.}
\item
  熟练掌握 NI ELVIS III 实验箱及 MeasurementsLive 软件仪器的使用方法。
\item
  能够独立完成硬件连线,特别是信号源、滤波器、RC
  网络及乘法器之间的连接,并确保公共端接地(GND)。
\item
  结合实验现象与理论知识,分析方波的频谱构成、一阶系统的充放电特性以及欠采样导致的信号失真原因。
\end{enumerate}

\section{总体方案}

针对设计内容,本设计采用``信号激励---系统处理---信号观测''的总体架构,依托
NI ELVIS III 平台实现。

\begin{enumerate}
\def\labelenumi{\arabic{enumi}.}
\item
  \textbf{信号获取与产生}: 使用计算机端的 MeasurementsLive
  软件控制函数信号发生器。

  \begin{itemize}
  \item
    \textbf{实验一}:产生频率 \(100\text{Hz}\)、幅值 \(2\text{Vpp}\)
    的周期方波 。
  \item
    \textbf{实验二}:产生 \(50\text{Hz}\) 的方波(占空比
    \(80\%\))和窄脉冲(占空比 \(5\%\)) 。
  \item
    \textbf{实验三}:产生 \(1000\text{Hz}\)
    正弦波作为被采样信号,以及不同频率的方波作为采样脉冲 {[}cite:
    65{]}。
  \end{itemize}
\item
  \textbf{信号处理(系统实现)}:

  \begin{itemize}
  \item
    \textbf{频谱分析}:信号接入板载``TUNEABLE
    LPF''模块,通过物理旋钮调节截止频率 \(f_c\)。
  \item
    \textbf{系统响应}:信号接入``RC NETWORK''模块,构建一阶动态电路。
  \item
    \textbf{采样实现}:利用``MULTIPLIER''模块将模拟信号与脉冲信号相乘,模拟脉冲调幅(PAM)过程。
  \end{itemize}
\item
  \textbf{信号观测}: 利用虚拟示波器(Oscilloscope)的通道
  1(CH1)监测输入信号,通道
  2(CH2)监测系统输出信号,通过对比输入输出波形的差异来分析系统特性。
\end{enumerate}

\section{方法原理}

\subsection{周期信号的频谱分析(傅里叶级数)}

根据傅里叶级数理论,满足狄利克雷条件的周期信号可以分解为直流分量与无穷多个谐波分量之和。对于本实验中的周期方波(无直流分量),其展开式仅包含奇次谐波:

\[x(t) = \frac{4A}{\pi} \sum_{n=1,3,5,\cdots}^{\infty} \frac{1}{n} \sin(n\omega_0 t)\]

低通滤波器(LPF)允许低频信号通过,衰减高频信号。当 LPF
的截止频率从低向高调节时,输出信号将依次包含基波、三次谐波、五次谐波等,时域波形将从正弦波逐渐趋近于方波,但在跳变沿会出现吉布斯现象(Gibbs
Phenomenon)。

\subsection{一阶系统时域响应}

RC 电路是典型的一阶线性时不变(LTI)系统。其电容电压 \(u_c(t)\)与输入电压
\(u_s(t)\) 的关系由一阶微分方程描述:

\[RC \frac{du_c(t)}{dt} + u_c(t) = u_s(t)\]

其中 \(\tau = RC\) 为时间常数。

\begin{itemize}
    \item \textbf{阶跃响应}:当输入为阶跃信号时,电容充电,输出按 \(1 - e^{-t/\tau}\) 规律上升。
    \item \textbf{冲激响应}:当输入为冲激信号时,电容瞬间充电后通过电阻放电,输出按 \(e^{-t/\tau}\) 规律衰减。
\end{itemize}

\subsection{采样定理与混叠}

奈奎斯特采样定理指出,为了无失真地恢复模拟信号,采样频率 \(f_s\)必须大于信号最高频率
\(f_{max}\) 的两倍,即:

\[f_s > 2f_{max}\]

本实验中 \(f_{max} = 1000\text{Hz}\)。

\begin{itemize}
    \item 当 \(f_s = 4000\text{Hz}\) 时,满足采样定理,信号可恢复。
    \item 当 \(f_s = 500\text{Hz}\) 时,\(f_s < 2f_{max}\),发生频谱混叠(Aliasing)。高频信号折叠到低频段\(|f_s - f_{signal}| = |500 - 1000| = 500\text{Hz}\),导致恢复出的信号频率错误。
\end{itemize}

\section{性能分析}

\subsection{Python 仿真分析}
本部分展示基于 Python 的理论仿真结果,用于指导后续的硬件实验。

\subsubsection{滤波器对波形合成的影响}

在实验一中,我们观测了不同截止频率下的输出波形。

\begin{itemize}
\item
  \textbf{低截止频率状态}: 当截止频率较低时,滤波器滤除了
  \(300\text{Hz}\)
  以上的高次谐波,输出仅剩基波。如图 1 所示,输出波形平滑,非常接近标准正弦波。
  
  \begin{figure}[H]
  \centering
  \includegraphics[width=0.8\textwidth]{exp1/exp1_low_cutoff.png}
  \caption{低截止频率下的滤波器输出(仅基波)}
  \end{figure}
\item
  \textbf{高截止频率状态}: 随着截止频率升高,更多奇次谐波通过。如图 2
  所示,输出波形趋近方波,但在上升沿和下降沿出现了明显的振荡(过冲),验证了吉布斯现象。

  \begin{figure}[H]
  \centering
  \includegraphics[width=0.8\textwidth]{exp1/exp1_high_cutoff.png}
  \caption{高截止频率下的滤波器输出(吉布斯现象)}
  \end{figure}
\end{itemize}

\subsubsection{RC 电路的动态响应特性}

在实验二中,我们对比了不同输入信号下的系统响应。

\begin{itemize}
\item
  \textbf{阶跃响应}: 输入为 \(50\text{Hz}\)
  方波时,电容电压不能突变,呈现指数上升和下降的特征。
  
  \begin{figure}[H]
  \centering
  \includegraphics[width=0.8\textwidth]{exp2/exp2_step_response.png}
  \caption{RC 电路阶跃响应}
  \end{figure}
\item
  \textbf{冲激响应}: 输入为窄脉冲时,电容瞬间充电后缓慢放电。
  
  \begin{figure}[H]
  \centering
  \includegraphics[width=0.8\textwidth]{exp2/exp2_impulse_response.png}
  \caption{RC 电路冲激响应}
  \end{figure}
\item
  \textbf{正弦稳态响应}: 输入为 \(1000\text{Hz}\)
  正弦波时,系统输出仍为正弦波,但幅度和相位发生变化。
  
  \begin{figure}[H]
  \centering
  \includegraphics[width=0.8\textwidth]{exp2/exp2_sine_response.png}
  \caption{RC 电路正弦稳态响应}
  \end{figure}
\end{itemize}

\subsubsection{采样频率与信号恢复}

在实验三中,验证了采样定理。

\begin{itemize}
\item
  \textbf{正常采样 (\(f_s = 4000\text{Hz}\))}:
  满足采样定理,恢复后的波形清晰还原了 \(1000\text{Hz}\) 正弦波。
  
  \begin{figure}[H]
  \centering
  \includegraphics[width=0.8\textwidth]{exp3/exp3_sampling_normal.png}
  \caption{4000Hz 采样下的信号恢复}
  \end{figure}
\item
  \textbf{欠采样 (\(f_s = 500\text{Hz}\))}:
  不满足采样定理,发生严重混叠。如图 7 所示,恢复出的信号频率变为
  \(500\text{Hz}\),且幅度衰减,无法代表原始信号。
  
  \begin{figure}[H]
  \centering
  \includegraphics[width=0.8\textwidth]{exp3/exp3_sampling_aliased.png}
  \caption{500Hz 采样下的混叠现象}
  \end{figure}
\end{itemize}

\subsection{硬件实测分析}
本部分展示在 NI ELVIS III 平台上采集的实际波形。

\subsubsection{滤波器实验实测}
(此处插入示波器截屏:低截止频率 vs 高截止频率)

\subsubsection{RC 电路实测}
(此处插入示波器截屏:方波响应 vs 窄脉冲响应)

\subsubsection{采样定理实测}
(此处插入示波器截屏:4kHz 采样 vs 500Hz 采样)

\section{主要问题及解决方法}

本设计过程中,遇到的主要问题及解决措施包括:

\begin{enumerate}
\def\labelenumi{\arabic{enumi}.}
\item
  \textbf{接地不良导致波形噪声大}:

  \begin{itemize}
  \item
    \textbf{问题}:初始连接时示波器波形杂波多,基准不稳。
  \item
    \textbf{解决}:检查发现部分 GND
    线未连接。严格按照实验要求,将所有模块的黑色数据线接地后,波形恢复清晰。
  \end{itemize}
\item
  \textbf{混叠波形观测困难}:

  \begin{itemize}
  \item
    \textbf{问题}:\(500\text{Hz}\) 采样时,示波器波形闪烁。
  \item
    \textbf{解决}:调整示波器时基(Time/Div)和触发电平(Trigger),捕捉到了稳定的低频混叠波形。
  \end{itemize}
\end{enumerate}

\section{工程应用潜力分析}

本设计验证的信号处理理论在工程中应用广泛: 1.
\textbf{通信滤波}:傅里叶级数与滤波原理是频分复用技术的基础,用于提取特定信道信号。
2.
\textbf{数字信号处理}:采样定理是模数转换(ADC)的核心。本实验展示的混叠现象强调了在
ADC 前端必须加装抗混叠滤波器的重要性。 3. \textbf{电路控制}:RC
电路的响应特性广泛应用于去抖动电路及 PID 控制器的积分/微分环节设计。



\section{附录:仿真代码}

以下为生成上述波形图的 Python 代码(文件位于项目根目录 \texttt{python/generate\_waves.py}):

\begin{lstlisting}
import numpy as np
import matplotlib.pyplot as plt
import os

# 路径配置:图片保存在上级目录的 images 文件夹中
BASE_IMAGES_DIR = os.path.join(os.path.dirname(os.path.dirname(os.path.abspath(__file__))), 'images')

# ... (其余代码与 python/generate_waves.py 一致)
\end{lstlisting}


\end{document}
